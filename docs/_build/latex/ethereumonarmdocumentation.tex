%% Generated by Sphinx.
\def\sphinxdocclass{report}
\documentclass[letterpaper,10pt,english]{sphinxmanual}
\ifdefined\pdfpxdimen
   \let\sphinxpxdimen\pdfpxdimen\else\newdimen\sphinxpxdimen
\fi \sphinxpxdimen=.75bp\relax

\PassOptionsToPackage{warn}{textcomp}
\usepackage[utf8]{inputenc}
\ifdefined\DeclareUnicodeCharacter
% support both utf8 and utf8x syntaxes
  \ifdefined\DeclareUnicodeCharacterAsOptional
    \def\sphinxDUC#1{\DeclareUnicodeCharacter{"#1}}
  \else
    \let\sphinxDUC\DeclareUnicodeCharacter
  \fi
  \sphinxDUC{00A0}{\nobreakspace}
  \sphinxDUC{2500}{\sphinxunichar{2500}}
  \sphinxDUC{2502}{\sphinxunichar{2502}}
  \sphinxDUC{2514}{\sphinxunichar{2514}}
  \sphinxDUC{251C}{\sphinxunichar{251C}}
  \sphinxDUC{2572}{\textbackslash}
\fi
\usepackage{cmap}
\usepackage[T1]{fontenc}
\usepackage{amsmath,amssymb,amstext}
\usepackage{babel}



\usepackage{times}
\expandafter\ifx\csname T@LGR\endcsname\relax
\else
% LGR was declared as font encoding
  \substitutefont{LGR}{\rmdefault}{cmr}
  \substitutefont{LGR}{\sfdefault}{cmss}
  \substitutefont{LGR}{\ttdefault}{cmtt}
\fi
\expandafter\ifx\csname T@X2\endcsname\relax
  \expandafter\ifx\csname T@T2A\endcsname\relax
  \else
  % T2A was declared as font encoding
    \substitutefont{T2A}{\rmdefault}{cmr}
    \substitutefont{T2A}{\sfdefault}{cmss}
    \substitutefont{T2A}{\ttdefault}{cmtt}
  \fi
\else
% X2 was declared as font encoding
  \substitutefont{X2}{\rmdefault}{cmr}
  \substitutefont{X2}{\sfdefault}{cmss}
  \substitutefont{X2}{\ttdefault}{cmtt}
\fi


\usepackage[Bjarne]{fncychap}
\usepackage{sphinx}

\fvset{fontsize=\small}
\usepackage{geometry}


% Include hyperref last.
\usepackage{hyperref}
% Fix anchor placement for figures with captions.
\usepackage{hypcap}% it must be loaded after hyperref.
% Set up styles of URL: it should be placed after hyperref.
\urlstyle{same}

\addto\captionsenglish{\renewcommand{\contentsname}{Contents:}}

\usepackage{sphinxmessages}
\setcounter{tocdepth}{1}



\title{Ethereum on ARM documentation}
\date{Jan 16, 2021}
\release{0.0.1}
\author{Diego Losada \textless{}dlosada@ethereumonarm.com\textgreater{}, Fernando Collado \textless{}fcollado@ethereumonarm.com\textgreater{}}
\newcommand{\sphinxlogo}{\vbox{}}
\renewcommand{\releasename}{Release}
\makeindex
\begin{document}

\pagestyle{empty}
\sphinxmaketitle
\pagestyle{plain}
\sphinxtableofcontents
\pagestyle{normal}
\phantomsection\label{\detokenize{index::doc}}


This is the official Ethereum on ARM documentation.

If you want to check our Git repositories:
\begin{itemize}
\item {} 
\sphinxhref{https://github.com/diglos/pi-gen/}{Pi Gen} Raspberry Pi 4 repository (forked from official Raspbian OS)

\item {} 
\sphinxhref{https://github.com/diglos/userpatches/}{User Patches} Other ARM devices repository (forked from Armbian)

\end{itemize}


\chapter{What is Ethereum on ARM}
\label{\detokenize{index:what-is-ethereum-on-arm}}
Ethereum on ARM is a custom Linux image
that runs Ethereum clients as a boot service
and automatically turns a Raspberry Pi 4 and other
devices into a fullEthereum 1.0 / 2.0 node.


\chapter{Quick Start Guide}
\label{\detokenize{index:quick-start-guide}}
If you know about Ethereum and you have already run an Ethereum 1.0
or 2.0 node you can jump to our quick start guide to get your
Raspberry Pi 4 up and running in no time.
\begin{itemize}
\item {} 
\sphinxstylestrong{What you need}:
{\hyperref[\detokenize{quick-guide/what-you-need::doc}]{\sphinxcrossref{\DUrole{doc}{What you need}}}}

\item {} 
\sphinxstylestrong{Download and install the image}:
{\hyperref[\detokenize{quick-guide/download-and-install::doc}]{\sphinxcrossref{\DUrole{doc}{Download and install the image}}}}

\item {} 
\sphinxstylestrong{Ethereum clients}:
{\hyperref[\detokenize{quick-guide/ethereum-clients::doc}]{\sphinxcrossref{\DUrole{doc}{Ethereum clients}}}}

\item {} 
\sphinxstylestrong{What’s next}:
{\hyperref[\detokenize{quick-guide/whats-next::doc}]{\sphinxcrossref{\DUrole{doc}{What’s next}}}}

\end{itemize}


\section{What you need}
\label{\detokenize{quick-guide/what-you-need:what-you-need}}\label{\detokenize{quick-guide/what-you-need::doc}}
We are currently focused on the Raspberry Pi 4 but we do support other devices such us NanoPC\sphinxhyphen{}T4 and
\begin{itemize}
\item {} 
Raspberry 4 (model B) 8GB

\item {} 
MicroSD Card (16 GB Class 10 minimum)

\item {} 
SSD USB 3.0 disk or an SSD with an USB to SATA case (see \DUrole{xref,std,std-doc}{Storage} section)

\item {} 
Power supply

\item {} 
Ethernet cable

\item {} 
Port forwarding (see clients for further info)

\item {} 
A case with heatsink and fan (Optional but strongly recommended)

\item {} 
USB keyboard, Monitor and HDMI cable (micro\sphinxhyphen{}HDMI) (Optional)

\end{itemize}


\section{Download and Install The Image}
\label{\detokenize{quick-guide/download-and-install:download-and-install-the-image}}\label{\detokenize{quick-guide/download-and-install::doc}}

\section{Ethereum Clients}
\label{\detokenize{quick-guide/ethereum-clients:ethereum-clients}}\label{\detokenize{quick-guide/ethereum-clients::doc}}

\section{What’s Next}
\label{\detokenize{quick-guide/whats-next:what-s-next}}\label{\detokenize{quick-guide/whats-next::doc}}\begin{itemize}
\item {} 
\DUrole{xref,std,std-ref}{genindex}

\item {} 
\DUrole{xref,std,std-ref}{modindex}

\item {} 
\DUrole{xref,std,std-ref}{search}

\end{itemize}



\renewcommand{\indexname}{Index}
\printindex
\end{document}